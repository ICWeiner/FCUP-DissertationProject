
%-------------------------------------------------------------------------
%	QUOTATION PAGE
%-------------------------------------------------------------------------
%\quotepage{Matt Smith as \emph{The Doctor}, written by Matthew Graham}
%{
%	I am and always will be the optimist, the hoper of far-flung hopes and the
%	dreamer of \newline improbable dreams
%}

%-------------------------------------------------------------------------
%	DEDICATORY
%-------------------------------------------------------------------------

%\begin{dedicatory}
%	Dedicated to (optional) 
%\end{dedicatory}

%-------------------------------------------------------------------------
%	ACKNOWLEDGEMENTS PAGE
%-------------------------------------------------------------------------
\addtocontents{toc}{\protect\setcounter{tocdepth}{-1}}
\begin{acknowledgements}

Acknowledge ALL the people! ඞ

\end{acknowledgements}
\addtocontents{toc}{\protect\setcounter{tocdepth}{3}}
%\addvspacetoc{0.3cm} % Add a gap in the Contents, for aesthetics


%-------------------------------------------------------------------------
%	ABSTRACT PAGE (PORTUGUESE)
%-------------------------------------------------------------------------
\addtocontents{toc}{\protect\setcounter{tocdepth}{-1}}
\begin{abstract}[
	thesistitle={Sistema para Avaliações Práticas de Administração de Redes },
	title={Resumo},
	degree={Mestrado em Engenharia de Redes e Sistemas Informáticos},
	nameconnector={por},
        keywordsname={Palavras-chave},
        keywords={física (keywords em português)}]
\begin{otherlanguage}{portuguese}

Este tese é sobre alguma coisa



\end{otherlanguage}
\end{abstract}
\addtocontents{toc}{\protect\setcounter{tocdepth}{3}}
%-------------------------------------------------------------------------
%	ABSTRACT PAGE
%-------------------------------------------------------------------------
\addtocontents{toc}{\protect\setcounter{tocdepth}{-1}}
\begin{abstract}

This thesis is about something, I guess.

\end{abstract}
\addtocontents{toc}{\protect\setcounter{tocdepth}{3}}

%-------------------------------------------------------------------------
%	LIST OF CONTENTS/FIGURES/TABLES
%-------------------------------------------------------------------------

\addtocontents{toc}{\protect\setcounter{tocdepth}{-1}}

\tableofcontents % Write out the Table of Contents

\addtocontents{toc}{\protect\setcounter{tocdepth}{3}}
\addvspacetoc{0.3cm}

%\listoftables % Write out the List of Tables

\listoffigures % Write out the List of Figures



%\addvspacetoc{0.3cm}

%-------------------------------------------------------------------------
%	PHYSICAL CONSTANTS/OTHER DEFINITIONS
%-------------------------------------------------------------------------

%\begin{listofcontants}
%	\const{My little ponny test of magical rainbow}{$mn/mp$}
%    {$2.997\ 924\ 58\times10^{8}\ \mbox{ms}^{-\mbox{s}}$}
%   \const{Vaccuum permeability test of magical rainbow for a specific case of
%   condensed matter physics}
%   {$\epsilon_0$}{$2.997\ 924\ 58\times10^{8}\ \mbox{ms}^{-\mbox{s}}$}
%	\const{Speed of Light test of magical rainbow}{$c$}
%    {$2.997\ 924\ 58\times10^{8}\ \mbox{ms}^{-\mbox{s}}$}
%\end{listofcontants}


%-------------------------------------------------------------------------
%	SYMBOLS
%-------------------------------------------------------------------------

%\begin{listofsymbols}
%	\symb{$F_{\mu\nu}$}{Maxwell tensor}{F}
%	\symb{$a$}{distance}{m}
%	\\
%	\symb{$\omega$}{angular frequency}{rads$^{-1}$}
%\end{listofsymbols}


%-------------------------------------------------------------------------
%	NOTATION
%-------------------------------------------------------------------------

% \newcommand\notationname{Notation and Conventions}
% \addtotoc{\notationname}
% \fancyhead[LO]{\textsc{\notationname}}

% \input{Notation}



%-------------------------------------------------------------------------
%	ABBREVIATIONS
%-------------------------------------------------------------------------

\newacronym{cs}{CS}{Computer Science}

\newacronym{dcc}{DCC}{Department of Computer Science}

\newacronym{gns3}{GNS3}{Graphical Network Simulator-3}

\newacronym{rest}{REST}{Representational State Transfer}

\newacronym{api}{API}{Application Programming Interface}

\newacronym{qemu}{QEMU}{Quick Emulator}

\newacronym{iou}{IOU}{IOS on Unix}

\newacronym{ios}{IOS}{Internetworking Operating System}

\newacronym{vpcs}{VPCS}{Virtual PC Simulator}

\newacronym{vm}{VM}{Virtual Machine}

\newacronym{http}{HTTP}{Hypertext Transfer Protocol}

\newacronym{json}{JSON}{JavaScript Object Notation}

\newacronym{wsgi}{WSGI}{Web Server Gateway Interface}

\newacronym{asgi}{ASGI}{Asynchronous Server Gateway Interface}

\newacronym{oas}{OAS}{OpenAPI Specification}

\newacronym{pve}{Proxmox VE}{Proxmox Virtual Environment}

\newacronym{kvm}{KVM}{Kernel-based Virtual Machine}

\newacronym{lxc}{LXC}{Linux Containers}

\newacronym{jwt}{JWT}{JSON Web Token}

\newacronym{ldap}{LDAP}{Lightweight Directory Access Protocol}

\newacronym{ssh}{SSH}{Secure Shell}

\newacronym{lvm}{LVM}{Logical Volume Manager}

\newacronym{lvmt}{LVM-Thin}{LVM Thin Provisioning}

\newacronym{cow}{CoW}{Copy-on-Write}

\newacronym{orm}{ORM}{Object-Relational Mapping}

\newacronym{spice}{SPICE}{Simple Protocol for Independent Computing Environments}








\printglossary[type=\acronymtype,title={List of Abbreviations}]
