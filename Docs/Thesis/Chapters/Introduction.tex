% Chapter Template

% Main chapter title
%\chapter[toc version]{doc version}
\chapter{Introduction}

% Short version of the title for the header
%\chaptermark{version for header}

% Chapter Label
% For referencing this chapter elsewhere, use \ref{ChapterTemplate}
\label{ChapterIntroduction}

% Write text in here
% Use \subsection and \subsubsection to organize text

In today's digital age the need for qualified \ac{cs} professionals is growing.
The\ac{cs} field is vast and has many areas of expertise, one of which is network administration.
It is a crucial part of any organization, as it is responsible for the maintenance and management of 
the organization's network infrastructure.
Proper training for network administrators is crucial for preparing them for real-world situations.
One way to provide this training is through practical evaluations, allowing students to apply the knowledge they 
have acquired in a real-world scenario, helping them to develop the skills they will need in their future careers.

Creating a physical network environment for practical evaluations may be costly and challenging to scale for 
large student populations. 
Emulation and virtualization technologies can help to simplify and cost-effectively create practice environments 
for students. 
These technologies alone do not address the issue of manually reviewing a network topology's setup. 
Manually reviewing each student's network configuration can be time-consuming and prone to human error, rendering 
it challenging for their instructors. Automating the evaluation process may substantially alleviate the burden 
on educators and guarantee uniform and fair assessments.



\subsection{Aims and Objectives}
This dissertation continues the work of a previous student, who carried out research and first steps of development of a 
system for automated evaluation system for network topologies. The main goal is to design and implement a scalable system 
capable of automatically evaluating evaluating network topologies that make use of different vendors and device types.
The support for different vendors and device types is crucial, as it allows students to practice with a variety of
networking equipment, preparing them for the real-world scenarios they will face in their future careers.
Automating the evaluation process will help educators dedicate more time to other tasks such as supporting students, and
would also provide a more consistent and fair evaluation, eliminating the possibility of human error.

--- Talk about the end goal ---

The main steps of this project are as follows:

\begin{itemize}
    \item Study the bases for the system already developed
    \item Requirements gathering
    \item Identification of the main problems that need to be solved
    \item Proposal of solutions for these problems
    \item System design
    \item Implementation of a prototype
    \item Testing with volunteers to validate the system and identify possible limitations.
  \end{itemize}
