% Chapter Template

% Main chapter title
%\chapter[toc version]{doc version}
\chapter{Related Work}

% Short version of the title for the header
%\chaptermark{version for header}

% Chapter Label
% For referencing this chapter elsewhere, use \ref{ChapterTemplate}
\label{Chapter3RelatedWork}

% Write text in here
% Use \subsection and \subsubsection to organize text

This chapter focuses on placing the current project within the context of existing solutions and related work. The primary 
goal of this project is to develop a system capable of automatically evaluating network topologies by validating device 
configurations and executing tests across a virtual network.

While automated assessment systems are well established in the field of programming education—receiving student-submitted 
code and running it against predefined test cases—equivalent systems for network exercises are far less common. Tools like 
Mooshak and similar platforms have proven effective for evaluating programming assignments and are widely adopted in academic 
settings.

At first glance, adapting these approaches to network topologies might seem straightforward. However, network evaluation 
introduces unique challenges such as the need for per-student virtual environments, real-time communication with multiple 
devices, and stateful, distributed configurations. This chapter explores existing tools like Mooshak and Packet Tracer, 
highlighting their capabilities, limitations, and how this project builds upon or diverges from them.

\section{Programming Evaluation Systems}
    While not directly related, they are the main inspiration for this project. Programming evaluation systems are widely
    deployed in universities and other educational institutions. These systems receive, as input, code from students and 
    subsequently run tests on it, outputting a score and even being configurable to provide students the first test case that 
    they failed in, guiding students to the solution without handing it out.

    The main differentiator between these systems and the one proposed in this project is the ability to solve a network 
    exercise using multiple configurations across multiple devices, while programming evaluation systems will expect
    the same output every time, given the same input.
    
    Another key difference is the fact that programming evaluation systems dont always provide a working environment for the 
    students to test their code, owing to the fact that students might prefer to user their own development environment for 
    initial development and testing. This project aims to provide a working environment for students, as setting up a 
    networking lab can be a daunting task for students, especially when they are just starting out. By providing a pre-configured
    environment, students can focus on learning the concepts and skills they need to succeed in their studies, rather than
    spending time troubleshooting their setup.

    \subsection{Mooshak and lessons learned}

        In our context, in the\ac{dcc}, Mooshak is commonly deployed to be used in the context of classes, exams and even 
        programming contests.

        Mooshak is a web-based system for managing programming contests and also to act as an automatic judge of programming 
        contests \cite{Leal2003567}. It supports a variety of programming languages like Java, C, etc. Under each contest students 
        will find one more problem definitions each containing varying sets of test cases in input-output pairs. After submiting 
        their solution, the system will compile and run the code against the test cases giving a score based on the the amount of 
        test cases passed.

        Mooshak provides a structured approach to test coding and problem solving skills. It begins by offering a problem statement 
        coupled with an optional image and an example test case, normally in the form of input and expected output.
        Users can submit their proposed solution by uploading a file with their code. The system then evaluates the provided solution 
        against multiple pre-defined test cases, validating the output against the know-good output, and giving feedback in the form of 
        a score based on the number of test cases passed. The system may also be configured to have time and/or memory constraints, 
        to ensure that temporal and spatial complexity are also taken into account.
    
        All of these, serve to provide a thorough evaluation of the student's solution, which can help guide a student to better
        their coding and problem solving skills.

        The system can also differentiate between differing types of errors, such as not giving the expected output, poorly 
        formatted output, failure to compile or even exceeding the time limits.
        Mooshak also includes some features designed to drive competition between students, like a real time leaderboard and
        the ability to have more than 100\% of the score for a given contest.

        The system however is not without its limitations as it uses plain text files for its test cases and validates the output 
        of student's code character by character, which can lead to false negatives if the output is not formatted exactly as
        expected.

\section{Cisco Packet Tracer}

    Cisco Packet Tracer is a network \textbf{simulation} tool developed by Cisco Systems, widely used in academic environments 
    to teach networking concepts and prepare students for certifications such as the Cisco Certified Network Associate (CCNA). 
    It offers a visual interface for building and simulating virtual network topologies using a variety of Cisco devices, 
    including routers, switches, and end devices.

    While Packet Tracer is highly accessible and effective for introducing networking fundamentals, it is a closed-source, 
    proprietary tool limited to simulating Cisco hardware and\ac{ios} features. Its functionality is optimized for teaching 
    purposes rather than for flexibility, extensibility, or integration into larger automated workflows.

    In contrast, this project aim to allow for a more realistic and extensible lab environment. The use of real operating 
    systems and support of a wide range of vendor platforms for routers and switches, aswell as Linux-based virtual machines 
    is highly desirable. This allows for a more realistic experience, as students will be able to work with the same tools and 
    operating systems that they will encounter in real-world scenarios.

    Therefore, while Cisco Packet Tracer remains a valuable educational tool, the needs of this project called for a more 
    flexible and open architecture.