% Chapter Template

% Main chapter title
%\chapter[toc version]{doc version}
\chapter{Introduction}

% Short version of the title for the header
%\chaptermark{version for header}

% Chapter Label
% For referencing this chapter elsewhere, use \ref{ChapterTemplate}
\label{Chapter1Introduction}

% Write text in here
% Use \subsection and \subsubsection to organize text

This chapter contextualizes challenges, outlining the limitations of current pedagogical tools and processes while framing the necessity 
for an automated, scalable solution. By dissecting the shortcomings of existing platforms to manual assessment burdens—we lay the groundwork 
for a system capable of aiding network administration education.


\section{Problem Statement}

The digital transformation sweeping across industries has created unprecedented demand for skilled computer science professionals.
While most, if not all,\ac{cs} specializations face growing needs, network administration remains a foundational requirement. This persistent demand 
reflects the networks's critical role as infrastructure supporting all digital systems.

Modern organizations require professionals who can design, configure, and troubleshoot increasingly complex network environments. 
Effective education must therefore bridge theoretical knowledge with practical implementation, particularly through hands-on evaluations 
that simulate real-world scenarios. Yet current assessment methods fail to meet these needs at scale, creating a growing gap between academic 
preparation and professional requirements.

\subsection{Need for Automated Evaluation}

Practical evaluations are essential to prepare students for real-world challenges, enabling them to apply 
theoretical knowledge and develop hands-on skills. However, available evaluation methods face significant limitations.

Creating a physical network environment for practical evaluations is sure to be costly and challenging to scale for 
large student populations. 
While emulation and virtualization technologies offer cost-effective alternatives for creating flexible practice environments, 
they lack built-in automated assessment capabilities. 
Instructors must manually review each student's network topology configuration—a process that is:

\begin{itemize}
  \item \textbf{Time-consuming}: Manual checks grow linearly with class size.
  \item \textbf{Error-prone}: Human reviewers may overlook misconfigurations.
  \item \textbf{Inconsistent}: Subjective grading criteria lead to unfair assessments.
\end{itemize}

Automating evaluations would reduce instructor workload, freeing time for student support aswell as ensure consistent, objective grading 
and simultaneously enable immediate feedback for learners.

\subsection{Limitations of Current Solutions}

Existing approaches to network education suffer from two critical gaps:

\begin{enumerate}
    \item \textbf{Single-vendor focus}: Most tools (e.g., Cisco's packet tracer) are designed for specific vendor ecosystems, failing 
    to prepare students for heterogeneous real-world networks where multi-vendor interoperability is essential.
        \item \textbf{Missing evaluation component}: Some education institutions' curricula, such as our case in\ac{dcc}, forgo practical 
        assessments entirely, which means:
    \begin{itemize}
        \item Students complete exercises without validation
        \item No measurable feedback on configuration skills
        \item Graduates enter industry with possible gaps in their knowledge
    \end{itemize}
\end{enumerate}


\section{Aims and Objectives}

Building upon the foundational work of \citet{santos2024} in automated network topology evaluation, this project has the following technical objectives:

\begin{enumerate}
    \item \textbf{Develop a prototypical automated evaluation environment}:
    \begin{itemize}
        \item Create a system for assessing network administration exercises without manual intervention
        \item Support multi-vendor device configurations (Cisco, Juniper, Linux-based)
        \item Validate both connectivity and configuration compliance
    \end{itemize}

    \item \textbf{Implement a cohesive back-end system}:
    \begin{itemize}
        \item Transform loose components from \citet{santos2024} into a unified system
        \item Develop capabilities to coordinate between:
        \begin{itemize}
            \item Infrastructure provisioning
            \item Network emulation 
            \item Evaluation automation
        \end{itemize}
    \end{itemize}
\end{enumerate}

