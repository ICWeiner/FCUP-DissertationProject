% Chapter Template

% Main chapter title
%\chapter[toc version]{doc version}
\chapter{System Architecture \& Design}

% Short version of the title for the header
%\chaptermark{version for header}

% Chapter Label
% For referencing this chapter elsewhere, use \ref{ChapterTemplate}
\label{ChapterSystemArchitectureDesign}

% Write text in here
% Use \subsection and \subsubsection to organize text

Delivering reliable, scalable automated assessment of virtualized networks requires a system built on solid principles of 
virtualization and automation. This chapter outlines the architecture of the proposed system, detailing the key components 
and how they interact to enable seamless evaluation of student-submitted network exercises.

The system is designed to provide each student with a working environment where custom network topologies can be 
deployed, configured, and tested. To achieve this, the platform integrates several technologies—such as \ac{gns3} for network 
emulation, \ac{pve} for virtualization, and Nornir for configuration testing—alongside an asynchronous web-based \ac{api} layer for 
user interaction and system communications.

This section provides a high-level overview of the system, the rationale behind its design choices, and the fundamental 
components that make up its architecture.

\section{System Architecture Overview}
The system architecture is designed to provide a robust and scalable solution for the automated assessment of network
topologies. The architecture is divided into several key components, each responsible for a specific aspect of the system's
functionality. The main components of the system architecture are as follows:

\begin{itemize}
    \item \textbf{Web App:} The web app serves as the main interface for users to interact with the system. It provides 
    endpoints for evaluation, creation  and viewing available exercises. The app is designed to be asynchronous where possible, 
    allowing for efficient handling of multiple requests simultaneously.
    
    \item \textbf{Proxmox VE:} \ac{pve} is responsible for creating and managing \ac{vm}s that host the network devices used in 
    the exercises. This layer interacts with the web app to create and manage the \ac{vm}s based on the creation of new 
    exercises or students. All communication with \ac{pve} is done asynchronously through the Proxmox \ac{rest} \ac{api}, 
    which allows for efficient communication, keeping the web app responsive, while also keeping the components decoupled. 
    
    \item \textbf{GNS3:} \ac{gns3} is used to emulate all the components of the virtual networks to be configured by students, 
    using various types of virtualization detailed earlier. Communication with \ac{gns3} is done through the \ac{gns3} 
    \ac{rest} \ac{api} \unsure{synchronously (will it stay this way)} by the web app, due to a lack of robustness of the \ac{gns3} \ac{api}.

    \item \textbf{Nornir:} This automation framework is used for validating device configurations. It connects to the 
    virtualized devices, executes commands, and compares the output to expected results to determine correctness.
\end{itemize}

\section{Component Breakdown}

\subsection{Web Application}
The web application serves as the primary interface through which users interact with the system. It is built using the FastAPI 
framework and follows an asynchronous-first, modular architecture that scalable interactions with other system components.

The application exposes a \ac{rest} \ac{api} that supports endpoints for user authentication, exercise creation, virtual 
machine management, and configuration validation. It acts as the coordinator for the entire system, triggering operations in 
\ac{pve}, \ac{pve}, and Nornir based on user actions.

Wherever possible, asynchronous I/O is employed to prevent blocking during operations such as \ac{api} calls to Proxmox.
Multiprocessing is also utilized to handle configuration validation. This keeps the system responsive and performant, 
especially when handling multiple simultaneous requests from different users.

Internally, the application is designed to be stateless and maintain minimal runtime state.  Most essential information—such 
as user accounts, defined exercises, and student-to-VM mappings—is persisted in a relational database rather than stored in 
memory. Configuration values such as API tokens, base URLs, and database credentials are injected via environment variables 
to decouple deployment-specific settings from the application code. This design improves reliability, supports concurrent 
usage, and enables horizontal scalability if deployed across multiple instances. 

To ensure maintainability and modularity, interactions with external services like \ac{pve} and \ac{gns3} are isolated in 
dedicated modules. These serve as abstraction layers between the application logic and third-party \ac{api}s, exposing clean, 
reusable interfaces while hiding low-level implementation details. For example, Proxmox-related operations such as \ac{vm} 
creation and deletion are handled in a separate module (e.g services/proxmox.py), as are all \ac{gns3}-related tasks. This 
separation of concerns improves the structure of the codebase and simplifies future maintainability by being more readable.

To help with development and testing, the application automatically generates OpenAPI-compliant documentation, allowing 
developers to explore and interact with available endpoints. This self-documenting behavior streamlines integration 
testing and encourages a more agile development process.

Finally, to safeguard user data and infrastructure control points, the application enforces secure authentication mechanisms 
using \ac{jwt} ensuring that only authorized users can trigger actions on shared resources. \unsure{JWT doesnt quite work yet}

\subsection{Proxmox Virtual Environment}
\ac{pve} functions as the virtualization backbone of the system, enabling the creation and management of Linux-based \ac{vm}s 
used in the automated assessment workflow. Each \ac{vm} runs a lightweight Linux operating system with a dedicated 
\ac{gns3}-server instance, providing a self-contained environment for deploying and configuring virtual networks and their 
components.

The lifecycle of a \ac{vm} in the system begins when a new exercise is created. At this point, the platform clones a 
pre-configured base template \ac{vm} available in \ac{pve}. This \ac{vm} is then started and automatically configured: the 
provided \ac{gns3} project file is imported, and a sequence of user-defined commands is executed within the \ac{gns3} 
environment. Once the setup is finalized, the configured \ac{vm} is converted into a new template \ac{vm} that is tailored to 
that exercise.

From this template, a “work” \ac{vm} is created for every active student. These \ac{vm}s are exact clones of the configured 
template and serve as the student's personal lab environment for a given exercise. Each student thus receives a consistent 
starting point for the exercise while working within an private instance of the network topology.

By default, these \ac{vm}s are not strictly isolated from each other at the network level. However, isolation can be 
enforced by dynamically applying firewall rules that restrict each student \ac{vm}'s network access to a single designated 
validation host—the same host that also runs the web application at the current stage of development. This approach ensures 
a secure and controlled assessment environment while retaining the flexibility to scale or adjust network policies as needed.

The \ac{pve} infrastructure is utilized with scalability in mind. The use of linked clones and storage-efficient backing 
filesystems, in this case LVM-thin, allows the system to rapidly provision VMs while minimizing storage usage. Load 
balancing across multiple \ac{pve} nodes is also possible, although the current implementation assumes a single-node 
deployment for simplicity.

All \ac{pve}-related operations—such as cloning, starting, templating, and deletion—are fully automated and triggered by the 
web application. Under normal operation, no manual intervention using the \ac{pve} web UI or shell utilities is required; 
such intervention is only necessary when the system's error handling mechanisms detect failures that cannot be automatically 
resolved. To securely execute these operations, the application authenticates to the \ac{pve} \ac{api} using token-based 
authentication. The required credentials and configuration parameters are securely injected via environment variables, while 
the time limited token is stored in memory, ensuring that only authorized and properly configured processes can interact 
with the \ac{pve} infrastructure.
\subsection{GNS3 Network Emulator}

\subsection{Nornir Automation Framework}

\subsection{Storage and Data Model (Maybe)}
% If your architecture has a DB or persistent state model, a section here works great
